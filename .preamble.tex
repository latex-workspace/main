\documentclass[12pt,a4paper,oneside]{article}
%
\usepackage[utf8]{inputenc}%gestisce caratteri UTF8
\usepackage[italian]{babel}%gestisce hyphenation
\usepackage[colorlinks]{hyperref}%gestisce collegamenti ipertestuali e hypelinkage
\usepackage{soulutf8}%sottolineature con line break
\usepackage{tabularx}%tabelle più sofisticate
\usepackage{graphics}%per importare immagini
\usepackage[table,xcdraw]{xcolor}%per usare colori
\usepackage{listings}%per listare pezzi di codice (c++ o altro)
\usepackage{amsmath}%per cose matematiche (es. allign)
\usepackage{mathtools}%per underbracket
\usepackage{amssymb}%per simboli tipo strettamente incluso
\usepackage{tikz}%per disegnare
\usepackage{pgfplots}%per fare grafici a barre e plottare funzioni
\usepackage{pgf-pie}%per fare grafici a torta
\usepackage[margin=1in, marginparwidth=50pt]{geometry}%per settare margini e spaziature
\usepackage{forest}%per disegnare alberi
\usepackage{etoolbox}%per sistemare stile (renderlo come amsart pur usando amsbook con capitoli)
\usepackage{tocloft}%per liste di comandi e teoremi
\usepackage{float}%per posizionamento tabelle
\usetikzlibrary{er, positioning,patterns}
\usepackage{ccicons}
\usepackage{booktabs}
\usepackage{titlesec}% for custom titles
\usepackage[nameinlink, italian]{cleveref}

\usepackage{mathrsfs}

\usepackage{tcolorbox}%per box colorati
\tcbuselibrary{skins}
\tcbuselibrary{breakable}

\usepackage[ruled, vlined]{algorithm2e}
\usepackage{subcaption}

% \tcbset{shield externalize}
\renewcommand{\labelitemi}{$\circ$}

\definecolor{mutedblue}{RGB}{72,70,120}
\definecolor{mutedcyan}{RGB}{55, 126, 126}
\definecolor{mutedred}{RGB}{167, 60, 40}
\definecolor{mutedpurple}{RGB}{128 60 136}
\definecolor{mutedgreen}{RGB}{55 126 34}
\definecolor{muteddarkgreen}{RGB}{108 119 48}

\hypersetup{
	colorlinks=true,
	linkcolor=mutedblue,
	citecolor=mutedgreen,
	filecolor=mutedred,
	urlcolor=mutedcyan,
}



%stili tikz
\tikzset{
	blackdot/.style={
			circle,fill,inner sep=1.5pt, anchor = center
		},
	whitedot/.style={
			circle,fill=white, draw, inner sep=1.5pt, anchor = center
		}
}
%SETTAGGI PRELIMINARI DEI VARI PACCHETTI/LAYOUT
%-----------------------------------------------------------------------------------
\setlength{\parindent}{0pt}%toglie indent paragrafo


\tikzset{Thin_lines/.style={line width=0.5pt}}%crea stile linee sottili per grafici a torta

\pgfplotstableset{col sep = tab}%dichiara separatore colonna in tabelle, per plottare i grafici a barre

\pagestyle{plain}%setta stile pagine: non viene inserito nulla di default (ad es. a piè di pagina)

\renewcommand{\underline}[1]{\ul{#1}}%sottolineatura con pacchetto ulem: permettte di usare shortcut di vscode

%\renewcommand\labelitemi{--}%primo livello di elenco puntato è un trattino invece che punto

\pgfplotsset{compat=newest}%setta versione alla più recente, evita problemi di compatibilità



%%%%%%%%%%%%%%%%%%%%%%%%%%%%%%%%%%%%%%%%%%%%%%%%%%%%%%%%%%
% Custom titles
%%%%%%%%%%%%%%%%%%%%%%%%%%%%%%%%%%%%%%%%%%%%%%%%%%%%%%%%%%
\newcommand{\sectioncolorbox}[1]
{%
	{%
			\Large\sffamily
			\tcbox[size=title,colback=mutedblue!50, on line]{\color{white}\thesection\quad}\quad \color{mutedblue}\underline{#1}
		}
}

\newcommand{\subsectioncolorbox}[1]
{%
	{%
			\large\sffamily
			\tcbox[size=small, colback=mutedblue!50, on line]{\color{white}\thesubsection\quad}\quad \color{mutedblue}\underline{#1}
		}
}

\newcommand{\subsubsectioncolorbox}[1]
{%
	{%
			\sffamily\color{mutedblue}
			\tcbox[size=fbox,colback=mutedblue!50, on line]{\color{white}\thesubsubsection\quad}
			\; \underline{#1}
		}
}

\titleformat
{\section}
[block]
{}
{}
{0pt}
{\sectioncolorbox}
[]

\titleformat
{\subsection}
[block]
{}
{}
{0pt}
{\subsectioncolorbox}
[]

\titleformat
{\subsubsection}
[block]
{}
{}
{0pt}
{\subsubsectioncolorbox}
[]



\newcommand{\sfblue}[1]{
	{\sffamily\textcolor{mutedblue}{#1}}
}

\newcommand{\sfred}[1]{
	{\sffamily\textcolor{mutedred}{#1}}
}


%%%%%%%%%%%%%%%%%%%%%%%%%%%%%%%%%%%%%%%%%%%%%%%%%%%%%%%%%%
% Algorithm preudocode
%%%%%%%%%%%%%%%%%%%%%%%%%%%%%%%%%%%%%%%%%%%%%%%%%%%%%%%%%%
\SetKwProg{Fn}{}{:}{}
\SetKw{And}{and}
\SetKw{Or}{or}
\SetKw{Downto}{downto}
\SetKw{Print}{print}
\SetKw{KwTo}{to}
\SetKw{KwFrom}{from}
\SetKw{KwTrue}{true}
\SetKw{KwNot}{not}
\SetKw{KwFalse}{false}


\SetKwFunction{Push}{push}
\SetKwFunction{Top}{top}
\SetKwFunction{Pop}{pop}


\SetKwData{Int}{int}
\SetKwData{Bool}{bool}
\SetKwData{Void}{void}
\SetKwData{Stack}{Stack}
\SetKwData{Set}{Set}
\SetKwData{Item}{Item}
\SetKwData{Graph}{Graph}
\SetKwData{Node}{Node}
\SetKwData{Queue}{Queue}
\SetKwData{PriorityQueue}{PriorityQueue}

\SetKwComment{Comment}{$\triangleright$\ }{}

\RestyleAlgo{plain}

%RENDE STILE CAPITOLI UGUALI AD AMSART PUR MANTENENDO I CAPITOLI
%-----------------------------------------------------------------------------------
%\renewcommand{\contentsname}{\centering Indice}%Cambia nome che appare in cima a indice

%\makeatletter
%\numberwithin{section}{chapter}
%\def\@secnumfont{\mdseries}
%\def\section{\@startsection{section}{1}%
% \z@{.7\linespacing\@plus\linespacing}{.5\linespacing}%
%{\normalfont\scshape\centering}}
%\def\subsection{\@startsection{subsection}{2}%
%\z@{.5\linespacing\@plus.7\linespacing}{-.5em}%
%{\normalfont\bfseries}}

%\patchcmd{\@thm}{\let\thm@indent\indent}{\let\thm@indent\noindent}{}{}
%\patchcmd{\@thm}{\thm@headfont{\scshape}}{\thm@headfont{\bfseries}}{}{}

%\makeatother

%\newtheorem{thm}{Theorem}[section]

%%%%%%%%%%%%%%%%%%%%%%%%%%%%%%%%%%%%%%%%%%%%%%%%%%%%%%%%%%
%  definisce liste di forrmule, di comandi e di teoremi  %
%%%%%%%%%%%%%%%%%%%%%%%%%%%%%%%%%%%%%%%%%%%%%%%%%%%%%%%%%%
%%%%%%%%%%%%%%%%
%%  Definizione
%%%%%%%%%%%%%%%%
% \newsavebox{\defbox}
% \newcounter{defcount}
% \newcommand{\deflist}{\Large Definizioni}
% \newlistof{defs}{def}{\deflist}
%
% \newenvironment{definizione}[1]{%
% 	\refstepcounter{defcount}%
% 	\def\deftitle{#1}%
% 	\begin{lrbox}{\defbox}%
% 		\begin{minipage}{\dimexpr\linewidth-2\tabcolsep\relax}%
% 			}{%
% 		\end{minipage}%
% 	\end{lrbox}%
% 	\noindent\vskip3mm
% 	\begin{tabular}{p{\dimexpr\linewidth-2\tabcolsep\relax}}
% 		\toprule
% 		\textbf{Definizione \arabic{defcount}:} \textit{\deftitle} \\
% 		\midrule
% 		\usebox{\defbox}                                           \\
% 		\bottomrule
% 	\end{tabular}%
% 	\addcontentsline{def}{defs}{\protect\numberline{\thedefcount}\deftitle}
% 	\noindent\vskip3mm
% }
% %%%%%%%%%%%%%%%%
% %  Teorema
% %%%%%%%%%%%%%%%%
% \newsavebox{\theoremsbox}
% \newcounter{theoremscount}
% \newcommand{\theoremlist}{\Large Teoremi e Assiomi}
% \newlistof{theorems}{the}{\theoremlist}
%
% \newenvironment{teorema}[1]{%
% 	\refstepcounter{theoremscount}%
% 	\def\teorematitle{#1}%
% 	\begin{lrbox}{\theoremsbox}%
% 		\begin{minipage}{\dimexpr\linewidth-2\tabcolsep\relax}%
% 			}{%
% 		\end{minipage}%
% 	\end{lrbox}%
% 	\noindent\vskip3mm
% 	\begin{tabular}{p{\dimexpr\linewidth-2\tabcolsep\relax}}
% 		\toprule
% 		\textbf{Teorema \arabic{theoremscount}:} \textit{\teorematitle} \\
% 		\midrule
% 		\usebox{\theoremsbox}                                           \\
% 		\bottomrule
% 	\end{tabular}%
% 	\addcontentsline{the}{theorems}{\protect\numberline{\thetheoremscount}\teorematitle}
% 	\noindent\vskip3mm
% }
%
% %%%%%%%%%%%%%%%%
% %  Assioma
% %%%%%%%%%%%%%%%%
% \newsavebox{\assiomabox}
% \newenvironment{assioma}[1]{%
% 	\refstepcounter{theoremscount}%
% 	\def\assiomatitle{#1}%
% 	\begin{lrbox}{\assiomabox}%
% 		\begin{minipage}{\dimexpr\linewidth-2\tabcolsep\relax}%
% 			}{%
% 		\end{minipage}%
% 	\end{lrbox}%
% 	\noindent\vskip3mm
% 	\begin{tabular}{p{\dimexpr\linewidth-2\tabcolsep\relax}}
% 		\toprule
% 		\textbf{Assioma \arabic{theoremscount}:} \textit{\assiomatitle} \\
% 		\midrule
% 		\usebox{\assiomabox}                                            \\
% 		\bottomrule
% 	\end{tabular}%
% 	\addcontentsline{the}{theorems}{\protect\numberline{\thetheoremscount}\assiomatitle}
% 	\noindent\vskip3mm
% }
%
% %%%%%%%%%%%%%%%%
% %  Formula
% %%%%%%%%%%%%%%%%
% \newsavebox{\formulabox}
% \newcounter{formulascount}
% \newcommand{\formulalist}{\Large Formule}
% \newlistof{formulas}{for}{\formulalist}
%
% \newenvironment{formula}[1]{%
% 	\refstepcounter{formulascount}%
% 	\def\formulatitle{#1}%
% 	\begin{lrbox}{\formulabox}%
% 		\begin{minipage}{\dimexpr\linewidth-2\tabcolsep\relax}%
% 			}{%
% 		\end{minipage}%
% 	\end{lrbox}%
% 	\noindent\vskip3mm
% 	\begin{tabular}{p{\dimexpr\linewidth-2\tabcolsep\relax}}
% 		\toprule
% 		\textbf{Formula \arabic{formulascount}:} \textit{\formulatitle} \\
% 		\midrule
% 		\usebox{\formulabox}                                            \\
% 		\bottomrule
% 	\end{tabular}%
% 	\addcontentsline{for}{formulas}{\protect\numberline{\theformulascount}\formulatitle}
% 	\noindent\vskip3mm
% }
% %%%%%%%%%%%%%%%%
% %  Esercizio
% %%%%%%%%%%%%%%%%
% \newsavebox{\eserciziobox}
% \newcounter{exercisescount}
% \newcommand{\exerciselist}{\Large Esercizi}
% \newlistof{exercises}{exe}{\exerciselist}
%
% \newenvironment{esercizio}[1]{%
% 	\refstepcounter{exercisescount}%
% 	\def\eserciziotitle{#1}%
% 	\begin{lrbox}{\eserciziobox}%
% 		\begin{minipage}{\dimexpr\linewidth-2\tabcolsep\relax}%
% 			}{%
% 		\end{minipage}%
% 	\end{lrbox}%
% 	\noindent\vskip3mm
% 	\begin{tabular}{p{\dimexpr\linewidth-2\tabcolsep\relax}}
% 		\toprule
% 		\textbf{Esercizio \arabic{exercisescount}:} \textit{\eserciziotitle} \\
% 		\midrule
% 		\usebox{\eserciziobox}                                               \\
% 		\bottomrule
% 	\end{tabular}%
% 	\addcontentsline{exe}{exercises}{\protect\numberline{\theexercisescount}\eserciziotitle}
% 	\noindent\vskip3mm
% }


\tcbset{
	boxrule=1pt,
}
%%%%%%%%%%%%%%%%
%  Definizione
%%%%%%%%%%%%%%%%
\newsavebox{\defbox}
\newcounter{defcount}
\newcommand{\deflist}{\Large Definizioni}
\newlistof{defs}{def}{\deflist}

\newenvironment{definizione}[1]{%

	\refstepcounter{defcount}%
	\def\deftitle{#1}%
	\noindent\vskip3mm
	\begin{tcolorbox}[colback=mutedred!5, boxrule=1pt, colbacktitle = mutedred!40,
			title={\textbf{Definizione \arabic{defcount}:} \textit{\deftitle}}]

		}{%
	\end{tcolorbox}
	\addcontentsline{def}{defs}{\protect\numberline{\thedefcount}\deftitle}
}

%%%%%%%%%%%%%%%%
%  Teorema
%%%%%%%%%%%%%%%%
\newsavebox{\theoremsbox}
\newcounter{theoremscount}
\newcommand{\theoremlist}{\Large Teoremi e Assiomi}
\newlistof{theorems}{the}{\theoremlist}

\newenvironment{teorema}[1]{%
	\refstepcounter{theoremscount}%
	\def\teorematitle{#1}%
	\noindent\vskip3mm
	\begin{tcolorbox}[colback=mutedblue!5, boxrule=1pt, colbacktitle=mutedblue!40,
			title={\textbf{Teorema \arabic{theoremscount}:} \textit{\teorematitle}}]
		}{%
	\end{tcolorbox}
	\addcontentsline{the}{theorems}{\protect\numberline{\thetheoremscount}\teorematitle}
}

%%%%%%%%%%%%%%%%
%  Assioma
%%%%%%%%%%%%%%%%
\newenvironment{assioma}[1]{%
	\refstepcounter{theoremscount}%
	\def\assiomatitle{#1}%
	\noindent\vskip3mm
	\begin{tcolorbox}[colback=mutedblue!5, boxrule=1pt, colbacktitle = mutedblue!40,
			title={\textbf{Assioma \arabic{theoremscount}:} \textit{\assiomatitle}}]
		}{%
	\end{tcolorbox}
	\addcontentsline{the}{theorems}{\protect\numberline{\thetheoremscount}\assiomatitle}
}

%%%%%%%%%%%%%%%%
%  Formula
%%%%%%%%%%%%%%%%
\newcounter{formulascount}
\newcommand{\formulalist}{\Large Formule}
\newlistof{formulas}{for}{\formulalist}

\newenvironment{formula}[1]{%
	\refstepcounter{formulascount}%
	\def\formulatitle{#1}%
	\noindent\vskip3mm
	\begin{tcolorbox}[colback=mutedpurple!5, boxrule=1pt, colbacktitle = mutedpurple!30,
			title={\textbf{Formula \arabic{formulascount}:} \textit{\formulatitle}}]
		}{%
	\end{tcolorbox}
	\addcontentsline{for}{formulas}{\protect\numberline{\theformulascount}\formulatitle}
}

%%%%%%%%%%%%%%%%
%  Esercizio
%%%%%%%%%%%%%%%%
\newcounter{exercisescount}
\newcommand{\exerciselist}{\Large Esercizi}
\newlistof{exercises}{exe}{\exerciselist}

\newenvironment{esercizio}[1]{%
	\refstepcounter{exercisescount}%
	\def\eserciziotitle{#1}%
	\noindent\vskip3mm
	\begin{tcolorbox}[colback=orange!5, boxrule=1pt, colbacktitle = orange!30,
			title={\textbf{Esercizio \arabic{exercisescount}:} \textit{\eserciziotitle}}]
		}{%
	\end{tcolorbox}
	\addcontentsline{exe}{exercises}{\protect\numberline{\theexercisescount}\eserciziotitle}
}

\newenvironment{esercizioL}[1]{%
	\refstepcounter{exercisescount}%
	\def\eserciziotitle{#1}%
	\noindent\vskip3mm
	\begin{tcolorbox}[enhanced, breakable, colback=orange!5, boxrule=1pt, colbacktitle = orange!30,
			title={\textbf{Esercizio \arabic{exercisescount}:} \textit{\eserciziotitle}}]
		}{%
	\end{tcolorbox}
	\addcontentsline{exe}{exercises}{\protect\numberline{\theexercisescount}\eserciziotitle}
}

%%%%%%%%%%%%%%%%
%  Algoritmo
%%%%%%%%%%%%%%%%
\newcounter{myalgorithmscount}
\newcommand{\algorithmlist}{\Large Algoritmi}
\newlistof{myalgorithms}{alg}{\algorithmlist}

\newenvironment{algoritmo}[1]{%
	\refstepcounter{myalgorithmscount}%
	\def\algoritmotitle{#1}%
	\noindent\vskip3mm
	\begin{tcolorbox}[colback=mutedpurple!5, boxrule=1pt, colbacktitle = mutedpurple!40,
			title={\textbf{Algoritmo \arabic{myalgorithmscount}:} \textit{\algoritmotitle}}]
		}{%
	\end{tcolorbox}
	\addcontentsline{alg}{myalgorithms}{\protect\numberline{\themyalgorithmscount}\algoritmotitle}
}

\newenvironment{algoritmo*}[1]{%
	\def\algoritmotitle{#1}%
	\noindent\vskip3mm
	\begin{tcolorbox}[colback=mutedpurple!5, boxrule=1pt, colbacktitle = mutedpurple!40,
			title={\textbf{Algoritmo:} \textit{\algoritmotitle}}]
		}{%
	\end{tcolorbox}
}


%%%%%%%%%%%%%%%%
%  Comandi
%%%%%%%%%%%%%%%%
\newcounter{commandscount}
\newcommand{\commandlist}{\Large Comandi}
\newlistof{commands}{com}{\commandlist}

\newcommand{\command}[4]{
	\vskip3mm
	\refstepcounter{commandscount}
	\begin{tabularx}{\textwidth}{@{}l>{\raggedleft\arraybackslash}X@{}}
		\toprule
		\arabic{commandscount}. \texttt{#1} \textcolor{gray}{\texttt{#2}} & #3 \\
		\multicolumn{2}{@{}l@{}}{
		\begin{minipage}{\textwidth}
			#4
		\end{minipage}
		}                                                                      \\
		\bottomrule
	\end{tabularx}
	\addcontentsline{com}{commands}{\protect\numberline{\thecommandscount}#1}\par
}


%%%%%%%%%%%%%%%%
%  Incomprensioni
%%%%%%%%%%%%%%%%
\newcounter{incomprensionicount}
\newcommand{\incomprensionilist}{\Large Incomprensioni}
\newlistof{incomprensioni}{inc}{\incomprensionilist}

\newcommand{\incomprensione}[1]{
	\refstepcounter{incomprensionicount}
	\addcontentsline{inc}{incomprensioni}{\protect\numberline{\theincomprensionicount}#1}\par
	\vskip3mm
	{\color{red}\hrulefill\textcolor{red}{\textit{Incomprensione - #1}}\hrulefill}
	\vskip3mm
}


% \newcounter{lawscount}
% \newcommand{\lawslist}{\Large Leggi}
% \newlistof{laws}{law}{\lawslist}
%
% \newcommand{\law}[2]{
% 	\refstepcounter{lawscount}
% 	\bigbox[\linewidth]{
% 		\textbf{Legge \arabic{lawscount}:} \textit{#1}\\
% 		#2}
% 	\addcontentsline{law}{laws}{\protect\numberline{\thelawscount}#1}\par
% }
%
%
%
% % Definition counter and list
% \newcounter{defcount}
% \newcommand{\deflist}{\Large Definizioni}
% \newlistof{defs}{def}{\deflist}
%
% \newcommand{\definizione}[2]{
% 	\refstepcounter{defcount}
% 	\vskip3mm
% 	\begin{tabularx}{\textwidth}{X}
% 		\toprule
% 		\textbf{Definizione \arabic{defcount}:} \textit{#1} \\
% 		\midrule
% 		#2                                                  \\
% 		\bottomrule
% 	\end{tabularx}
% 	\addcontentsline{def}{defs}{\protect\numberline{\thedefcount}#1}\par
% }
%
% % Theorem/Axiom counter and list
% \newcounter{theoremscount}
% \newcommand{\theoremlist}{\Large Teoremi e Assiomi}
% \newlistof{theorems}{the}{\theoremlist}
%
% \newcommand{\teorema}[2]{
% 	\refstepcounter{theoremscount}
% 	\vskip3mm
% 	\begin{tabularx}{\textwidth}{X}
% 		\toprule
% 		\textbf{Teorema \arabic{theoremscount}:} \textit{#1} \\
% 		\midrule
% 		#2                                                   \\
% 		\bottomrule
% 	\end{tabularx}
% 	\addcontentsline{the}{theorems}{\protect\numberline{\thetheoremscount}#1}\par
% }
%
% \newcommand{\assioma}[2]{
% 	\refstepcounter{theoremscount}
% 	\vskip3mm
% 	\begin{tabularx}{\textwidth}{X}
% 		\toprule
% 		\textbf{Assioma \arabic{theoremscount}:} \textit{#1} \\
% 		\midrule
% 		#2                                                   \\
% 		\bottomrule
% 	\end{tabularx}
% 	\addcontentsline{the}{theorems}{\protect\numberline{\thetheoremscount}#1}\par
% }
%
%
% % Formula counter and list
% \newcounter{formulascount}
% \newcommand{\formulalist}{\Large Formule}
% \newlistof{formulas}{for}{\formulalist}
%
% \newcommand{\formula}[2]{
% 	\refstepcounter{formulascount}
% 	\vskip3mm
% 	\begin{tabularx}{\textwidth}{X}
% 		\toprule
% 		\textbf{Formula \arabic{formulascount}:} \textit{#1} \\
% 		\midrule
% 		#2                                                   \\
% 		\bottomrule
% 	\end{tabularx}
% 	\addcontentsline{for}{formulas}{\protect\numberline{\theformulascount}#1}\par
% }
%
% % Esercizio counter and list
% \newcounter{exercisescount}
% \newcommand{\exerciselist}{\Large Esercizi}
% \newlistof{exercises}{exe}{\exerciselist}
%
% \newcommand{\esercizio}[2]{
% 	\refstepcounter{exercisescount}
% 	\vskip3mm
% 	\begin{tabularx}{\textwidth}{X}
% 		\toprule
% 		\textbf{Esercizio \arabic{exercisescount}:} \textit{#1} \\
% 		\midrule
% 		#2                                                      \\
% 		\bottomrule
% 	\end{tabularx}
% 	\addcontentsline{exe}{exercises}{\protect\numberline{\theexercisescount}#1}\par
% }
%
% \newcommand{\esercizioNB}[2]{
% 	\refstepcounter{exercisescount}
% 	\textbf{Esercizio \arabic{exercisescount}:} \textit{#1}\\
% 	#2
% 	\addcontentsline{exe}{exercises}{\protect\numberline{\theexercisescount}#1}\par
% }




%CREA ENVIRONMENT "CODE", PER LISTARE SNIPPET IN C++
%-----------------------------------------------------------------------------------
\newcounter{my_code_counter}
\setcounter{my_code_counter}{0}

\definecolor{pink}{RGB}{224, 110, 160}
\definecolor{comment}{RGB}{100,100,105}
\definecolor{string}{RGB}{241,125,109}
\definecolor{std_functions}{RGB}{155,114,204}
\definecolor{std_members}{RGB}{190,163,223}
\definecolor{functions}{RGB}{66, 140, 162}
\definecolor{numbers}{RGB}{195, 181, 112}
\definecolor{light}{rgb}{0.5, 0.5, 0.5}

\lstdefinelanguage{cpp}{
	classoffset = 1, keywordstyle = \color{pink}\textbf,
	morekeywords = {class, continue, default, enum, false, new, null, private, protected, public, static, true, void, return, do, for, if, else, switch, while, try, int, long, char, float, double, short, unsigned, auto, using, namespace, constexpr},
	%
	%
	classoffset = 2,keywordstyle = \color{std_functions},
	morekeywords = {begin, end, push_back, push, size, lenght, resize, empty, at, find, emplace, emplace_back, erase, now, count, cout, endl, sort},
	%
	%
	classoffset = 3,keywordstyle = \color{std_members},
	morekeywords = {std, chrono, high_resolution_clock, vector, list, unordered_set, unordered_map},
	%
	%
	classoffset = 4,keywordstyle = \color{functions},
	morekeywords = {main, Polinomial, char_to_node, allowed_functions, allowed_sequences, contained, detect_exponential, get_priority_brackets, fix_syntax, checkSyntax},
	%
	%
	%
	%
	classoffset = 0,
	morecomment=[l]{//},
	morecomment=[s]{/*}{*/},
	morestring=[b]",
	morestring=[d]',
	stringstyle = \color{string},
	commentstyle = \color{comment}\textit
}

\lstset{
literate=
	{á}{{\'a}}1 {é}{{\'e}}1 {í}{{\'i}}1 {ó}{{\'o}}1 {ú}{{\'u}}1
{Á}{{\'A}}1 {É}{{\'E}}1 {Í}{{\'I}}1 {Ó}{{\'O}}1 {Ú}{{\'U}}1
{à}{{\`a}}1 {è}{{\`e}}1 {ì}{{\`i}}1 {ò}{{\`o}}1 {ù}{{\`u}}1
{À}{{\`A}}1 {È}{{\'E}}1 {Ì}{{\`I}}1 {Ò}{{\`O}}1 {Ù}{{\`U}}1
{ä}{{\"a}}1 {ë}{{\"e}}1 {ï}{{\"i}}1 {ö}{{\"o}}1 {ü}{{\"u}}1
{Ä}{{\"A}}1 {Ë}{{\"E}}1 {Ï}{{\"I}}1 {Ö}{{\"O}}1 {Ü}{{\"U}}1
{â}{{\^a}}1 {ê}{{\^e}}1 {î}{{\^i}}1 {ô}{{\^o}}1 {û}{{\^u}}1
{Â}{{\^A}}1 {Ê}{{\^E}}1 {Î}{{\^I}}1 {Ô}{{\^O}}1 {Û}{{\^U}}1
{œ}{{\oe}}1 {Œ}{{\OE}}1 {æ}{{\ae}}1 {Æ}{{\AE}}1 {ß}{{\ss}}1
{ű}{{\H{u}}}1 {Ű}{{\H{U}}}1 {ő}{{\H{o}}}1 {Ő}{{\H{O}}}1
{ç}{{\c c}}1 {Ç}{{\c C}}1 {ø}{{\o}}1 {å}{{\r a}}1 {Å}{{\r A}}1
{€}{{\euro}}1 {£}{{\pounds}}1,
extendedchars=true,
breakatwhitespace=false,         % sets if automatic breaks should only happen at whitespace
breaklines=true,                 % sets automatic line breaking
captionpos=t,                    % sets the caption-position to bottom 
firstnumber=1,                   % start line enumeration with line 1
frame=tblr,                        % adds a frame around the code
keepspaces=true,                 % keeps spaces in text, useful for keeping indentation of code (possibly needs columns=flexible)
language=cpp,                 % the language of the code
numbersep=5pt,                   % how far the line-numbers are from the code
showspaces=false,
showstringspaces=false,          % underline spaces within strings only
showtabs=false,                  % show tabs within strings adding particular underscores
stepnumber=1,                    % the step between two line-numbers. If it's 1, each line will be numbered
tabsize=2,
belowcaptionskip=10pt,
abovecaptionskip=10pt,
framesep=15pt,
numbers = left,
numberstyle=\tiny\color{light},
xleftmargin = 0.035\textwidth,
xrightmargin = 0.035\textwidth
}

\lstnewenvironment{code}[2]
{\nl
	\refstepcounter{my_code_counter}
	\label{#2}
	\qquad\textit{\rm{CODE SNIPPET\;\arabic{my_code_counter}\quad\quad #1}}
}
{\nl}




%COMANDI CARRARA
%-----------------------------------------------------------------------------------
\newcommand{\s}{\, {\rm{s}}}
\newcommand{\m}{\, {\rm{m}}}

\newcommand{\R}{\mathbb R}
\newcommand{\Q}{\mathbb Q}
\newcommand{\F}{\mathbb F}
\newcommand{\C}{\mathbb C}
\newcommand{\Z}{\mathbb Z}
\newcommand{\N}{\mathbb N}
\newcommand{\MCD}[2]{{\rm{MCD}} ( {#1},{#2} ) }
\newcommand{\mcm}[2]{{\rm{mcm}} ( {#1},{#2} )}
\newcommand{\Imm}{{\mathrm{Im}}}

%\renewcommand{\sin}[1]{  {\rm{sin}} {\left( {#1} \right) }}
\renewcommand{\aa}{\alpha}
\renewcommand{\AA}{\alpha}
\newcommand{\tg}{{\rm{tg}}}
\newcommand{\cotg}{{\rm{cotg}}}
%\newcommand{\sec}{{\rm{sec}}}
\newcommand{\cosec}{{\rm{cosec}}}
\newcommand{\asin}{{\rm{arcsin}}}
\newcommand{\acos}{{\rm{arccos}}}
\newcommand{\atg}{{\rm{arctg}}}
\newcommand{\dx}{\, {\rm{d}}x}
\newcommand{\dt}{\, {\rm{d}}t}
\newcommand{\e}{\, {\rm{e}}}

\newcommand{\intestazione}{
	\title[]{Liceo \emph{B. Russell} -- a.s. 2021/2022 -- classe V C}}

\newcommand{\studente}{Studente: $\underline{\phantom{cccccccccccccccccccccccccccccccccccccccccccccccccccccccccccccccccccccccc}}$}

\newcommand{\titolo}[1]{\vskip3mm\begin{center}{\bf{{#1}}}\end{center}}
\newcommand{\titolosc}[1]{\vskip3mm\begin{center}{\sc{{#1}}}\end{center}\vskip3mm}

\newcommand{\fine}{
	%\vskip1mm
	\begin{flushright}
		$\Box $
	\end{flushright}
}

%Per esercizi in \rm
%\newcommand{\esercizio}[1]
%{\begin{Ese}
%{\rm{
%{#1}
%}}
%\end{Ese}
%}

\newcommand{\quesito}[1]
{\begin{ques}
		{\rm{
				{#1}
			}}
	\end{ques}
}

\newcommand{\problema}[1]
{\begin{probl}
		{\rm{
				{#1}
			}}
	\end{probl}
}


%FRECCE, BOX E ALTRO
%-----------------------------------------------------------------------------------
%freccia a destra
\newcommand{\rarr}{$\rightarrow$}

%freccia verso il basso
\newcommand{\darr}{
	\vskip0mm
	\makebox[\textwidth][c]{$\Downarrow$}
	\vskip0mm
}

%box a larghezza fissa; comando aux
\newcommand{\fixedSizeTextbox}[2][\textwidth]{
	\noindent
	\vskip3mm
	\noindent\makebox[\textwidth][c]{
		\framebox{
			\begin{minipage}[c]{#1}
				\begin{center}
					\noindent#2
				\end{center}
			\end{minipage}
		}
	}
	\vskip3mm
}

\newcommand{\bigbox}[2][\linewidth]{
	\vskip3mm
	\noindent
	\framebox[#1]{
		\begin{minipage}[t]{0.97#1}
			#2
		\end{minipage}
	}
	\vskip3mm
}

%box a larghezza variabile; comando aux
\newcommand{\adaptiveTextbox}[1]{
	\noindent\vskip3mm
	\makebox[\textwidth][c]{
		\framebox{\noindent#1}
	}
	\vskip3mm
}

%comando che crea box orizzontale in base a larghezza contenuto
\newlength{\myfoo}
\newcommand{\textbox}[1]{\settowidth{\myfoo}{#1}%
	\ifnum\textwidth>\myfoo{\adaptiveTextbox{#1}}\else{\bigbox[\linewidth]{#1}}\fi
}

%va a capo, serve a poco
% \newcommand{\nl}{\hfill \\}

%cread \dotfill con punti che sono centrati verticalmente
\makeatletter
\newcommand\cdotfill{%
\leavevmode\cleaders\hb@xt@.44em{\hss$\cdot$\hss}\hfill\kern\z@
}

%riempie orizzontalmente di tratti
\newcommand{\tratti}{
	\vskip0mm
	\noindent
	\cdotfill
	\vskip0mm
}

%citazione letteraria
\newcommand{\cit}[2]{
	\vskip3mm
	\begin{flushleft}
		\begin{large}
			«
		\end{large}
	\end{flushleft}
	\begin{center}
		\begin{minipage}{0.8\textwidth}
			\textit{#1}
			\begin{flushright}
				\scriptsize{\textit{- #2}}
			\end{flushright}
		\end{minipage}
	\end{center}
	\begin{flushright}
		\large»
	\end{flushright}
}


\tikzstyle{albero_orizz}=[grow'=right,
edge from parent path={(\tikzparentnode.east) -- ++(0.5cm,0) |- (\tikzchildnode.west)}, growth parent anchor=east, child anchor=left, auto]


%formula in box centrara senza descrizione
\newcommand{\f}[1]{
	\vskip3mm
	\begin{center}
		\boxed{#1}
	\end{center}
}

%formula in box a sinistra con descrizione a destra
\newcommand{\Formula}[2]{
	\vskip0mm
	\boxed{#1}
	\hfill
	\begin{footnotesize}
		#2
	\end{footnotesize}
	\vskip5mm
}


%crea box in risalto con indicizzazione -- ref con numero: \ref -- ref con nume: \hyperlink
%se non viene indicato un label si può far riference con il suo numero
\newcounter{n_points}
\setcounter{n_points}{0}
\newcommand{\Point}[2][\arabic{n_points}]{
	\vskip3mm
	\refstepcounter{n_points}\label{#1}
	\hypertarget{#1}{
		\noindent\framebox{\textit{\arabic{n_points}. #2}}
	}\vskip3mm\noindent
}

%crea box in risalto senza indicizzazione
\newcommand{\point}[1]{
	\vskip3mm
	\hypertarget{#1}{
		\noindent\framebox{\textit{#1}}
	}
	\vskip3mm\noindent
}

%crea linea orizzontale
\newcommand{\hr}{\vskip3mm\noindent\hrule\vskip3mm}

%crea titoletto
\newcommand{\ter}[1]{
	\vskip10mm
	\noindent\hrulefill
	\framebox{#1}\hrulefill
	\vskip0mm
}

%definisce il simbolo uguale per definizione(quello che usa sempre il Moggio!)
\newcommand*{\defeq}{\stackrel{\text{def}}{=}}

%definisce tema di forest per fare alberi orizzontali con bracci quadrati
\forestset{
declare dimen={fork sep}{0.5em},
forked edge'/.style={
edge={rotate/.option=!parent.grow},
edge path'={(!u.parent anchor) -- ++(\forestoption{fork sep},0) |- (.child anchor)},
},
forked edge/.style={
		on invalid={fake}{!parent.parent anchor=children},
		child anchor=parent,
		forked edge',
	},
forked edges/.style={for nodewalk={#1}{forked edge}},
forked edges/.default=tree,
}

%definisce tema di forest per fare alberi orizzontali come alberi di dyrectory
\forestset{
declare dimen register=folder indent,
folder indent=.45em,
folder/.style={
parent anchor=-children last,
anchor=parent first,
calign=child,
calign primary child=1,
for children={
child anchor=parent,
anchor=parent first,
edge={rotate/.option=!parent.grow},
edge path'/.expanded={
([xshift=\forestregister{folder indent}]!u.parent anchor) |- (.child anchor)
},
},
after packing node={
		if n children=0{}{
				tempdiml=l_sep()-l("!1"),
				tempdims={-abs(max_s("","")-min_s("",""))-s_sep()},
				for children={
						l+=tempdiml,
						s+=tempdims()*(reversed()-0.5)*2,
					},
			},
	},
}
}
%
\newcommand{\placedate}[2]{\begin{center}
		\noindent\framebox{#1}\hrulefill\framebox{#2}
	\end{center}}
%
\newcommand{\Var}{\mathbb{V}ar}

\newcommand{\license}[1]{
	\begin{center}
		\noindent\textit{#1} is licensed under \href{https://creativecommons.org/licenses/by/4.0/?ref=chooser-v1}{\underline{CC BY 4.0}} \ccby.
		\vskip3mm
		\ccCopy \; 2023 \href{https://github.com/mattia-marini}{\underline{Mattia Marini}}
	\end{center}
}



\definecolor{mygray}{rgb}{0.5,0.5,0.5}
\definecolor{myred}{rgb}{0.869, 0.35, 0.32}
\definecolor{mygreen}{rgb}{0.659, 0.76, 0.53}
\definecolor{myyellow}{rgb}{0.824, 0.78, 0.55}
\definecolor{mypurp}{rgb}{0.58,0,0.82}
\definecolor{mypink}{rgb}{0.784, 0.549, 0.627}

\tikzset {
	multi attribute/.style = { attribute , double distance=1.5pt },
	derived attribute/.style = { attribute , dashed },
	total/.style = { double distance=1.5pt },
	every entity/.style = { draw=myred, fill=myred!20 },
	every attribute/.style = { draw=mypurp, fill=mypurp!20 },
	every relationship/.style = { draw=mygreen, fill=mygreen!20 }
}

\newcommand {\key} [1] {\underline{#1}}

\lstdefinelanguage{json}
{
	morestring=[b]",
	morestring=[d]'
}

\lstset{
	classoffset = 0,
	language=SQL,
	basicstyle={\small\ttfamily},
	belowskip=3mm,
	breakatwhitespace=true,
	breaklines=true,
	columns=flexible,
	framexleftmargin=0.25em,
	frameshape={}{yy}{}{}, %To remove to vertical lines on left, set `frameshape={}{}{}{}`
	commentstyle=\color{mygray},
	keywordstyle=\color{myred},
	numberstyle=\tiny\color{mypink},
	stringstyle=\color{mygreen},
	numbers=none, %If you want line numbers, set `numbers=left`
	showstringspaces=false,
	tabsize=3,
	xleftmargin =1em
}

\lstdefinestyle{SQL}{
	classoffset=0,
	morekeywords={PRAGMA, INTEGER, TEXT},
	classoffset = 1,
	morekeywords={INTEGER},
	keywordstyle=\color{myyellow}
}

\newcommand{\inputsql}[1]{%
	\lstinputlisting[
		classoffset=0,
		frame=none,
		language=SQL,
		style=SQL,
	]{#1}%
}

\endofdump

% pdftex -ini -jobname=".preamble" "&pdflatex" mylatexformat.ltx .preamble.tex
